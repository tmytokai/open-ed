\noindent \textbf{【出題意図】}

\noindent 1次元配列のメモリ空間内での配置のされ方とアドレスの調べ方を理解しているかどうかを確かめる問題である。

%%%%%%%%%%%%%%%%%%%%%%%%%%%%%%%%%%%
\vspace{1em}
\noindent \textbf{【重要事項】}

\medskip
\noindent ・ 配列はメモリ空間内で配列の各要素が長さ分だけ連続的につながっている領域として配置される。

\medskip
\noindent ・ 配列の先頭(要素の)アドレスは配列名だけで得られる。

\medskip
\noindent ・ a + i 又は \&a[i] で配列 a の i 番目の要素のアドレスを調べられる。

%%%%%%%%%%%%%%%%%%%%%%%%%%%%%%%%%%%
\vspace{1em}
\noindent \textbf{【解説】}

\noindent 配列が 0 番目から始まることと i の数字が何であるかに気をつければ答は求まる。

