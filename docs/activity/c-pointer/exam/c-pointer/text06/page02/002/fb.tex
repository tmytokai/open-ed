\noindent \textbf{【出題意図】}

\noindent 「配列渡し」によって元の配列の各要素の値を取得したり変更したり出来るかどうかを確かめる問題である。

%%%%%%%%%%%%%%%%%%%%%%%%%%%%%%%%%%%
\vspace{1em}
\noindent \textbf{【重要事項】}

\medskip
\noindent ・ 「配列渡し」は「ポインタ渡し」の一種でポインタ変数を引数として使う

\medskip
\noindent ・ 関数の中で引数を使いたい時は *(pa+i)記法 や pa[i] 記法の好きな方を使う

%%%%%%%%%%%%%%%%%%%%%%%%%%%%%%%%%%%
\vspace{1em}
\noindent \textbf{【解説】}

\noindent pa[i] の中身を変更すると main 関数の a[i] の中身も変わることから答が求まる。
