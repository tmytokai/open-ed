\documentclass[a4paper]{jarticle}

\def\texdir{../../../tex}
\makeatletter 

\pagestyle{plain}

\parindent = 1zw
\hoffset = -1in
\voffset = -1in
\topmargin = 15mm
\headheight = 0mm
\headsep = 0mm
\oddsidemargin = 15mm
\evensidemargin = 15mm
\textwidth = 210mm  
\textheight = 287mm
\addtolength{\textwidth}{-\oddsidemargin}
\addtolength{\textwidth}{-\evensidemargin}
\addtolength{\textheight}{-\topmargin}
\addtolength{\textheight}{-\headheight}
\addtolength{\textheight}{-\headsep}
\addtolength{\textheight}{-\footskip}

%%%% 演習ページ用マクロ %%%%

% ページ番号
\newcounter{c-page}
\setcounter{c-page}{0}

% 課題番号
\newcounter{c-prac}
\setcounter{c-prac}{0}

% 合計時間
\newcounter{practime}
\setcounter{practime}{0}

\newcommand{\prachead}[2]{%
\noindent アクティビティ: #1\par%
\noindent 学習項目: #2\par%
\noindent 演習時間: \ref{label:practime} 分\par%
\vspace{2zh}%
\noindent \textbf{\Large 演習}%
}

\newcommand{\pracpage}[1]{%
\stepcounter{c-page}%
\vspace{4zh}%
\noindent%
\textbf{\noindent \textbf{\Large \arabic{c-page}. #1}}%
}

\newcommand{\practeam}[1]{%
\stepcounter{c-prac}%
\vspace{2zh}%
\noindent%
\textbf{演習 \arabic{c-prac} (チーム, #1 分):}%
\advance \c@practime by #1%
}

\newcommand{\pracnote}[1]{%
\stepcounter{c-prac}%
\vspace{2zh}%
\noindent%
\textbf{演習 \arabic{c-prac} (個人ノート, #1 分):}%
\advance \c@practime by #1%
}

\newcommand{\pracpc}[1]{%
\stepcounter{c-prac}%
\vspace{2zh}%
\noindent%
\textbf{演習 \arabic{c-prac} (個人PC, #1 分):}%
\advance \c@practime by #1%
}

\newcommand{\savepractime}{%
\addtocounter{practime}{-1}%
\refstepcounter{practime}%
\label{label:practime}%
}

\makeatother 


\begin{document}

\pagehead{離散フーリエ変換 (DFT)}{[1] DFT と IDFT}{5}{高速フーリエ変換 (FFT)}

\bigskip
DFT / IDFT の最大の欠点は計算量が多くて時間がかかることです。具体的な DFT / IDFT の計算量(オーダー)は次の通りです。

\begin{framed}
\noindent\quad \textbf{DFT / IDFT の計算量(オーダー):}

\bigskip
$\textrm{N}$ を周期とした時

\[
\textrm{O}(\textrm{N}^2)
\]
\end{framed}

\medskip
そこで DFT を高速化した高速フーリエ変換( Fast Fourier Transform, FFT )、及び高速フーリエ逆変換( Inverse Fast Fourier Transform, IFFT )が考案されました。

\medskip
FFT / IFFT は周期 $\textrm{N}$ が $2$ の累乗(例えば $\textrm{N} = 2, 4, 8, 16, 32, 64, \cdots$ )でないと使えないという制限はありますが、次の様に計算量が劇的に減ります。もちろん計算結果は DFT / IDFT と同一になります。

\begin{framed}
\noindent\quad \textbf{FFT / IFFT の計算量(オーダー):}

\bigskip
$\textrm{N}$ を周期とした時(※)

\[
\textrm{O}(\textrm{N}\log\textrm{N})
\]

\bigskip
\noindent ※ ただし $\textrm{N}$ は $2$ の累乗(例えば $\textrm{N} = 2, 4, 8, 16, 32, 64, \cdots$ ) である必要がある。
\end{framed}

\medskip
$\textrm{N}$ が小さい時はそれ程計算速度は変わりませんが、大きくなるにつれて速度の差が大きくなっていきます。例えば $\textrm{N} = 1000$ の時は $300$ 倍位 FFT の方が速いです。従って世間一般的には DFT はあまり使われず、代わりに FFT が良く使われます。

\medskip
具体的な FFT / IFFT の計算方法についてはアクティビティの範囲外になりますし、フリーの FFT ライブラリが Web 上に沢山ありますので興味がある人は自分で調べて下さい。

\end{document}
