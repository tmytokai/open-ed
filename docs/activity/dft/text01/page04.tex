\documentclass[a4paper]{jarticle}

\def\texdir{../../../tex}
\makeatletter 

\pagestyle{plain}

\parindent = 1zw
\hoffset = -1in
\voffset = -1in
\topmargin = 15mm
\headheight = 0mm
\headsep = 0mm
\oddsidemargin = 15mm
\evensidemargin = 15mm
\textwidth = 210mm  
\textheight = 287mm
\addtolength{\textwidth}{-\oddsidemargin}
\addtolength{\textwidth}{-\evensidemargin}
\addtolength{\textheight}{-\topmargin}
\addtolength{\textheight}{-\headheight}
\addtolength{\textheight}{-\headsep}
\addtolength{\textheight}{-\footskip}

%%%% 演習ページ用マクロ %%%%

% ページ番号
\newcounter{c-page}
\setcounter{c-page}{0}

% 課題番号
\newcounter{c-prac}
\setcounter{c-prac}{0}

% 合計時間
\newcounter{practime}
\setcounter{practime}{0}

\newcommand{\prachead}[2]{%
\noindent アクティビティ: #1\par%
\noindent 学習項目: #2\par%
\noindent 演習時間: \ref{label:practime} 分\par%
\vspace{2zh}%
\noindent \textbf{\Large 演習}%
}

\newcommand{\pracpage}[1]{%
\stepcounter{c-page}%
\vspace{4zh}%
\noindent%
\textbf{\noindent \textbf{\Large \arabic{c-page}. #1}}%
}

\newcommand{\practeam}[1]{%
\stepcounter{c-prac}%
\vspace{2zh}%
\noindent%
\textbf{演習 \arabic{c-prac} (チーム, #1 分):}%
\advance \c@practime by #1%
}

\newcommand{\pracnote}[1]{%
\stepcounter{c-prac}%
\vspace{2zh}%
\noindent%
\textbf{演習 \arabic{c-prac} (個人ノート, #1 分):}%
\advance \c@practime by #1%
}

\newcommand{\pracpc}[1]{%
\stepcounter{c-prac}%
\vspace{2zh}%
\noindent%
\textbf{演習 \arabic{c-prac} (個人PC, #1 分):}%
\advance \c@practime by #1%
}

\newcommand{\savepractime}{%
\addtocounter{practime}{-1}%
\refstepcounter{practime}%
\label{label:practime}%
}

\makeatother 


\begin{document}

\pagehead{離散フーリエ変換 (DFT)}{[1] DFT と IDFT}{4}{DFTとIDFTの意味}

\bigskip
「複素フーリエ級数展開」のアクティビティで

\medskip
\noindent\textbf{複素フーリエ係数を求めると任意の周期性時間領域アナログ信号 $f(t)$ を直流成分、基本波、無限個の第$k$ 高調波の和に分解出来る}

\medskip
\noindent ことを学びました。これがいわゆる「フーリエの定理」です。

\medskip
一方、IDFTは複素フーリエ級数展開のディジタル版に相当しますので、やはり

\medskip
\noindent\textbf{DFT 係数を求めると任意の周期性時間領域ディジタル信号 $f[i]$ を直流成分、基本波、有限個(無限個で無いことに注意!)の第$k$ 高調波の和に分解出来る}

\medskip
\noindent ことを意味しています。

\medskip
以上のことを詳しく説明すると次の通りになります。
ただし周期 $\textrm{N}$ が奇数か偶数かによって最後の高調波の振幅の求め方が若干変わるので場合分けする必要があります。

\medskip
まず周期 $\textrm{N}$が奇数の場合です。

\begin{framed}
\noindent\quad \textbf{周期 $\textrm{N}$が奇数の場合:}

\[
f[i] = a_0 + \sum_{k=1}^{(\textrm{N}-1)/2}
\left \{
a_k \cdot \cos \left ( \frac{2\pi}{\textrm{T}_k} \cdot i + \phi_k \right )
\right \}
\]

\[
\textrm{T}_k = \frac{\textrm{N}}{k}
\]

\[
a_0 = \textrm{C}[0]
\]

\[
a_k = 2 \cdot |\textrm{C}[k]|
\]

\[
\phi_k = \angle \textrm{C}[k]
\]

\bigskip
\noindent\quad $f[i]$ ・・・ 周期 $\textrm{N}$ [点] (ただし奇数) の周期性時間領域ディジタル信号

\bigskip
\noindent\quad $\textrm{T}_k$・・・第 $k$ 高調波($k=1$の時は基本波)の周期、実数の\underline{定数}、単位は [点]

\bigskip
\noindent\quad $a_0$・・・直流成分、実数の\underline{定数}、範囲は実数全体、単位は扱う信号の種類による (ボルトとかアンペアとか度とか etc.)

\bigskip
\noindent\quad $a_k$・・・第 $k$ 高調波($k=1$の時は基本波)の振幅、実数の\underline{定数}、範囲は実数全体、単位は扱う信号の種類による (ボルトとかアンペアとか度とか etc.)

\bigskip
\noindent\quad $\phi_k$ ・・・第 $k$ 高調波($k=1$の時は基本波)の初期位相、実数の\underline{定数}、単位は [rad]

\bigskip
\noindent\quad $\textrm{C}[k]$ ・・・$k$ 番目のDFT係数
\end{framed}

\newpage
次は $\textrm{N}$が偶数の場合です。振幅の計算方法が他の高調波と異なるため最後の高調波が$\sum$の外に出ています。

\begin{framed}
\noindent\quad \textbf{周期 $\textrm{N}$が偶数の場合:}

\[
f[i] = a_0 + \sum_{k=1}^{\textrm{N}/2-1}
\left \{
a_k \cdot \cos \left ( \frac{2\pi}{\textrm{T}_k} \cdot i + \phi_k \right )
\right \}
+
a_{(\textrm{N}/2)} \cdot \cos \left ( \frac{2\pi}{\textrm{T}_{(\textrm{N}/2)}} \cdot i + \phi_{(\textrm{N}/2)} \right )
\]

\[
\textrm{T}_k = \frac{\textrm{N}}{k}
\]

\[
a_0 = \textrm{C}[0]
\]

\[
a_k = 2 \cdot |\textrm{C}[k]| 
\]

\[
\phi_k = \angle \textrm{C}[k]
\]

\[
\textrm{T}_{(\textrm{N}/2)} = \frac{\textrm{N}}{\textrm{N}/2} = 2
\]

\[
a_{(\textrm{N}/2)} = |\textrm{C}[\textrm{N}/2]|  
\]
\begin{center}
※ $a_k$と違って2倍しないことに注意!
\end{center}

\[
\phi_{(\textrm{N}/2)} = \angle \textrm{C}[\textrm{N}/2]
\]

\bigskip
\noindent\quad $f[i]$ ・・・ 周期 $\textrm{N}$ [点] (ただし偶数) の周期性時間領域ディジタル信号

\bigskip
\noindent\quad $\textrm{T}_k$・・・第 $k$ 高調波($k=1$の時は基本波)の周期、実数の\underline{定数}、単位は [点]

\bigskip
\noindent\quad $a_0$・・・直流成分、実数の\underline{定数}、範囲は実数全体、単位は扱う信号の種類による (ボルトとかアンペアとか度とか etc.)

\bigskip
\noindent\quad $a_k$・・・第 $k$ 高調波($k=1$の時は基本波)の振幅、実数の\underline{定数}、範囲は実数全体、単位は扱う信号の種類による (ボルトとかアンペアとか度とか etc.)

\bigskip
\noindent\quad $\phi_k$ ・・・第 $k$ 高調波($k=1$の時は基本波)の初期位相、実数の\underline{定数}、単位は [rad]

\bigskip
\noindent\quad $\textrm{T}_{(\textrm{N}/2)}$・・・第 $\textrm{N}/2$ 高調波の周期

\bigskip
\noindent\quad $a_{(\textrm{N}/2)}$・・・第 $\textrm{N}/2$ 高調波の振幅

\bigskip
\noindent\quad $\phi_{(\textrm{N}/2)}$ ・・・第 $\textrm{N}/2$ 高調波の初期位相

\bigskip
\noindent\quad $\textrm{C}[k]$ ・・・$k$ 番目のDFT係数
\end{framed}

\end{document}
