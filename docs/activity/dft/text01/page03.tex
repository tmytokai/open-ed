\documentclass[a4paper]{jarticle}

\def\texdir{../../../tex}
\makeatletter 

\pagestyle{plain}

\parindent = 1zw
\hoffset = -1in
\voffset = -1in
\topmargin = 15mm
\headheight = 0mm
\headsep = 0mm
\oddsidemargin = 15mm
\evensidemargin = 15mm
\textwidth = 210mm  
\textheight = 287mm
\addtolength{\textwidth}{-\oddsidemargin}
\addtolength{\textwidth}{-\evensidemargin}
\addtolength{\textheight}{-\topmargin}
\addtolength{\textheight}{-\headheight}
\addtolength{\textheight}{-\headsep}
\addtolength{\textheight}{-\footskip}

%%%% 演習ページ用マクロ %%%%

% ページ番号
\newcounter{c-page}
\setcounter{c-page}{0}

% 課題番号
\newcounter{c-prac}
\setcounter{c-prac}{0}

% 合計時間
\newcounter{practime}
\setcounter{practime}{0}

\newcommand{\prachead}[2]{%
\noindent アクティビティ: #1\par%
\noindent 学習項目: #2\par%
\noindent 演習時間: \ref{label:practime} 分\par%
\vspace{2zh}%
\noindent \textbf{\Large 演習}%
}

\newcommand{\pracpage}[1]{%
\stepcounter{c-page}%
\vspace{4zh}%
\noindent%
\textbf{\noindent \textbf{\Large \arabic{c-page}. #1}}%
}

\newcommand{\practeam}[1]{%
\stepcounter{c-prac}%
\vspace{2zh}%
\noindent%
\textbf{演習 \arabic{c-prac} (チーム, #1 分):}%
\advance \c@practime by #1%
}

\newcommand{\pracnote}[1]{%
\stepcounter{c-prac}%
\vspace{2zh}%
\noindent%
\textbf{演習 \arabic{c-prac} (個人ノート, #1 分):}%
\advance \c@practime by #1%
}

\newcommand{\pracpc}[1]{%
\stepcounter{c-prac}%
\vspace{2zh}%
\noindent%
\textbf{演習 \arabic{c-prac} (個人PC, #1 分):}%
\advance \c@practime by #1%
}

\newcommand{\savepractime}{%
\addtocounter{practime}{-1}%
\refstepcounter{practime}%
\label{label:practime}%
}

\makeatother 


\begin{document}

\pagehead{離散フーリエ変換 (DFT)}{[1] DFT と IDFT}{3}{プログラミング}

\bigskip
DFTの演算中には複素数が含まれていますので、C言語などでプログラミングしてDFT係数を求めるためには

\[
\textrm{C}[k] = \textrm{A}[k] + j \cdot \textrm{B}[k]
\]

\medskip
\noindent の様にDFT係数 $\textrm{C}[k]$ を実数成分 $\textrm{A}[k]$ と虚数成分 $\textrm{B}[k]$ に分けて求めると楽です。

\medskip
そこでオイラー公式を使ってDFTの演算中に含まれる複素数を $\sin$ と $\cos$ の和に分解してみると、 
$\textrm{A}[k]$ と $\textrm{B}[k]$ は次の様にして求められます。

\begin{framed}
\noindent\quad \textbf{DFTをプログラミングで求める}

\begin{align*}
\textrm{A}[k] &= \frac{1}{\textrm{N}} \sum_{i=0}^{\textrm{N}-1}
\left \{
f[i] \cdot \cos( k \cdot \frac{2\pi}{\textrm{N}} \cdot i ) 
\right \}
\ ,\  (k = 0, 1, 2, \cdots, \textrm{N}-1) \\
\textrm{B}[k] &= -\frac{1}{\textrm{N}} \sum_{i=0}^{\textrm{N}-1}
\left \{
f[i] \cdot \sin( k \cdot \frac{2\pi}{\textrm{N}} \cdot i )
\right \}
\ ,\  (k = 0, 1, 2, \cdots, \textrm{N}-1)
\end{align*}

\bigskip
\noindent\quad $f[i]$ ・・・ 周期 $\textrm{N}$ [点] の周期性ディジタル信号

\bigskip
\noindent\quad $\textrm{A}[k]$ ・・・$k$ 番目のDFT係数の実数成分、実数の\underline{定数}、個数は $\textrm{A}[0], \textrm{A}[1], \cdots, \textrm{A}[\textrm{N}-1]$ の $\textrm{N}$ 個 

\bigskip
\noindent\quad $\textrm{B}[k]$ ・・・$k$ 番目のDFT係数の虚数成分、実数の\underline{定数}、個数は $\textrm{B}[0], \textrm{B}[1], \cdots, \textrm{B}[\textrm{N}-1]$ の $\textrm{N}$ 個 
\end{framed}

\medskip
このようにすると DFT の演算中に複素数が含まれなくなるので、普通に外側が $k$、内側が $i$ の 2 重 for ループを作って $\textrm{A}[k]$ と $\textrm{B}[k]$ を求められます。なお \textrm{B}[k] は $\textrm{N}$ ではなく $-\textrm{N}$ で割る事に注意して下さい。

\medskip
また C 言語なら sqrt 関数と atan2 関数を使って

\begin{align*}
|\textrm{C}[k]| & = \textrm{sqrt}(\textrm{A}[k]*\textrm{A}[k] + \textrm{B}[k]*\textrm{B}[k]) \\
\angle \ \textrm{C}[k] & = \textrm{atan2}(\textrm{B}[k],\textrm{A}[k])
\end{align*}

\medskip
\noindent より DFT 係数 $\textrm{C}[k]$ の絶対値 $|\textrm{C}[k]|$ と偏角 $\angle \ \textrm{C}[k]$ も求まります。

\bigskip
同様に IDFT を $\textrm{A}[k]$ と $\textrm{B}[k]$ を使って書き直すと次のようになります。

\begin{framed}
\noindent\quad \textbf{IDFTをプログラミングで求める}

\[
f[i] = \sum_{k=0}^{\textrm{N}-1}
\left \{
\textrm{A}[k] \cdot \cos( k \cdot \frac{2\pi}{\textrm{N}} \cdot i )
- 
\textrm{B}[k] \cdot \sin( k \cdot \frac{2\pi}{\textrm{N}} \cdot i )
\right \}
\ ,\  (i = 0, 1, 2, \cdots, \textrm{N}-1)
\]

\bigskip
\noindent\quad $f[i]$ ・・・ 周期 $\textrm{N}$ [点] の周期性ディジタル信号

\bigskip
\noindent\quad $\textrm{A}[k]$ ・・・$k$ 番目のDFT係数の実数成分、実数の\underline{定数}、個数は $\textrm{A}[0], \textrm{A}[1], \cdots, \textrm{A}[\textrm{N}-1]$ の $\textrm{N}$ 個 

\bigskip
\noindent\quad $\textrm{B}[k]$ ・・・$k$ 番目のDFT係数の虚数成分、実数の\underline{定数}、個数は $\textrm{B}[0], \textrm{B}[1], \cdots, \textrm{B}[\textrm{N}-1]$ の $\textrm{N}$ 個 
\end{framed}

\medskip
この場合は外側が $i$、内側が $k$ の 2 重 for ループを作って $\textrm{A}[k]$ と $\textrm{B}[k]$ から $f[i]$ を復元できます。

\end{document}
