\noindent \textbf{【出題意図】}

\noindent 理論上は振幅とサンプリング周波数の値は無関係であることを確かめる問題である。

%%%%%%%%%%%%%%%%%%%%%%%%%%%%%%%%%%%
\vspace{1em}
\noindent \textbf{【重要事項】}

\bigskip
\noindent\quad 振幅 $a$ はボルトやアンペア・温度・音量などの物理量の大きさ・ボリュームを表し、扱う対象によって単位が変わる

\bigskip
\noindent\quad 振幅 $a$の値を変えるとグラフでは縦方向の大きさが変わる

\bigskip
\noindent\quad 振幅 $a$ がマイナスの場合は上下が反転したグラフになる

\bigskip
\noindent\quad 振幅 $a$ が $0$ の場合は$f[i]=\{\cdots,0,0,0,\cdots\}$ になる


%%%%%%%%%%%%%%%%%%%%%%%%%%%%%%%%%%%
\vspace{1em}
\noindent \textbf{【解説】}

\noindent 理論上は振幅とサンプリング周波数の値は無関係なので、振幅を変えてもサンプリング周波数は変化しない。
