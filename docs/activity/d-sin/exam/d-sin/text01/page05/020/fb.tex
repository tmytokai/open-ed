\noindent \textbf{【出題意図】}

\noindent 時間領域ディジタルサイン波 $f[i]$ の位相を反転させた式を求めることができるかどうかを確かめる問題である。

%%%%%%%%%%%%%%%%%%%%%%%%%%%%%%%%%%%
\vspace{1em}
\noindent \textbf{【重要事項】}

\bigskip
\noindent  初期位相 $\phi$ を $\pm\pi$ [rad] する事は「時間領域ディジタルサイン波の位相が反転している」と言って、元の初期位相 $0$ のディジタルサイン波が上下反転したグラフになる

\bigskip
\noindent  位相反転は振幅 $a$ の符号を反転させることと同じ意味


%%%%%%%%%%%%%%%%%%%%%%%%%%%%%%%%%%%
\vspace{1em}
\noindent \textbf{【解説】}

\noindent 元の式の位相に $\pm\pi$ [rad] を加えた式か、または振幅の符号を反転させた式が求める式となる。
