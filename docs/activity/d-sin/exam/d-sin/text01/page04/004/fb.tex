\noindent \textbf{【出題意図】}

\bigskip
\noindent 時間領域ディジタルサイン波の位相から平行移動方向と距離を求めることができるかどうかを確かめる問題である。

%%%%%%%%%%%%%%%%%%%%%%%%%%%%%%%%%%%
\vspace{1em}
\noindent \textbf{【重要事項】}

\noindent ・初期位相 $\phi$と進み・遅れの関係

\medskip
\begin{center}
\small
\begin{tabularx}{0.9\fbwidth}{|X|X|X|X|}
\hline
$\phi$ の符号&  位相が・・・   & 点で言い換えると・・・   & 並行移動方向と距離 \\
\hline
プラス &  $\phi$ (rad) 進んでいる & $\phi/(2\pi) \cdot \textrm{T}_d$ (点) 進んでいる & 左へ $\phi/(2\pi) \cdot \textrm{T}_d$ (点) \\
\hline
マイナス &  $|\phi|$ (rad) 遅れている & $|\phi|/(2\pi) \cdot \textrm{T}_d$ (点) 遅れている & 右へ $|\phi|/(2\pi) \cdot \textrm{T}_d$ (点) \\
\hline
\end{tabularx}
\end{center}

\bigskip

%%%%%%%%%%%%%%%%%%%%%%%%%%%%%%%%%%%
\vspace{1em}
\noindent \textbf{【解説】}

\bigskip
\noindent 重要事項の表より求まる。
