\noindent \textbf{【出題意図】}

\noindent 時間領域アナログサイン波の(角)周波数を変化させるとグラフがどの様に変化するかを確かめる問題である。

%%%%%%%%%%%%%%%%%%%%%%%%%%%%%%%%%%%
\vspace{1em}
\noindent \textbf{【重要事項】}

\[
f(t) = a \cdot \sin( w \cdot t + \phi )
\]
%
\noindent または
%
\[
f(t) = a \cdot \cos( w \cdot t + \phi )
\]

\medskip
\noindent を時間領域アナログサイン波と呼ぶ

\bigskip
\noindent $a$・・・振幅

\bigskip
\noindent $w$ ・・・角周波数 [rad/秒]

\bigskip
\noindent $\phi$ ・・・初期位相 [rad]

\bigskip
\noindent $t$ ・・・時刻 [秒]

\bigskip
$\textrm{T}$・・・周期 [秒]、$\textrm{T} = \frac{1}{f} = \frac{2\pi}{w}$

\bigskip
$w = 2\pi \cdot f = 2 \pi \cdot \frac{1}{\textrm{T}}$

\bigskip
$f = \frac{1}{\textrm{T}} = \frac{w}{2\pi}$

%%%%%%%%%%%%%%%%%%%%%%%%%%%%%%%%%%%
\vspace{1em}
\noindent \textbf{【解説】}

\noindent (角)周波数を変化させると横方向に伸縮する。
