\noindent \textbf{【出題意図】}

\noindent グラフから時間領域アナログサイン波の角周波数を求めることができるかどうかを確かめる問題である。

%%%%%%%%%%%%%%%%%%%%%%%%%%%%%%%%%%%
\vspace{1em}
\noindent \textbf{【重要事項】}

\[
f(t) = a \cdot \sin( w \cdot t + \phi )
\]
%
\noindent または
%
\[
f(t) = a \cdot \cos( w \cdot t + \phi )
\]

\medskip
\noindent を時間領域アナログサイン波と呼ぶ

\bigskip
\noindent $a$・・・振幅

\bigskip
\noindent $w$ ・・・角周波数 [rad/秒]

\bigskip
\noindent $\phi$ ・・・初期位相 [rad]

\bigskip
\noindent $t$ ・・・時刻 [秒]

\bigskip
$w = 2\pi f$

%%%%%%%%%%%%%%%%%%%%%%%%%%%%%%%%%%%
\vspace{1em}
\noindent \textbf{【解説】}

\noindent 波が $1$ [秒]に $2$ 回振動している、つまり$f=2$ [Hz]であることから、$w = 4\pi$ [rad/秒]が答えとなる。

