\noindent \textbf{【出題意図】}

\noindent 時間領域アナログサイン波を音として出力した時、周波数と音(音階)の高さの関係の知識を確かめる問題である。

%%%%%%%%%%%%%%%%%%%%%%%%%%%%%%%%%%%
\vspace{1em}
\noindent \textbf{【重要事項】}

\medskip
\noindent ・周波数 $f$ [Hz]を高くする → 音(音階)は高くなる

\medskip
\noindent ・周波数 $f$ [Hz]を低くする → 音(音階)は低くなる

\medskip
\noindent ・振幅 $a$ の絶対値を大きくする → 音量が上がる

\medskip
\noindent ・振幅 $a$ の絶対値を小さくする → 音量が下がる

\medskip
\noindent ・初期位相 $\phi$ を変える → 特に変化は無い様に聞こえる(個人差有り)

%%%%%%%%%%%%%%%%%%%%%%%%%%%%%%%%%%%
\vspace{1em}
\noindent \textbf{【解説】}

\noindent 今回の問は「$f$ [Hz]を高くしたとき」であるので、「音(音階)が高くなる」が答えとなる
