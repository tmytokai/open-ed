\noindent \textbf{【出題意図】}

\noindent 時間領域アナログサイン波を音として出力した時、周波数と音階の高さの関係の知識を確かめる問題である。

%%%%%%%%%%%%%%%%%%%%%%%%%%%%%%%%%%%
\vspace{1em}
\noindent \textbf{【重要事項】}

\medskip
\noindent ・周波数または角周波数を高くする → 音階は高くなる

\medskip
\noindent ・周波数または角周波数を高くする → 音階は低くなる

\medskip
\noindent ・振幅を大きくする → 音量が上がる

\medskip
\noindent ・振幅を小さくする → 音量が下がる

\medskip
\noindent ・初期位相を変える → 特に変化は無い様に聞こえる(個人差有り)

%%%%%%%%%%%%%%%%%%%%%%%%%%%%%%%%%%%
\vspace{1em}
\noindent \textbf{【解説】}

\noindent 今回の問は音階を「低く」したときであるので、「(角)周波数を低くする」が答えとなる
