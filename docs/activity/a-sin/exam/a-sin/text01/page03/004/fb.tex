\noindent \textbf{【出題意図】}

\noindent 交流電圧の実効値の知識を確かめる問題である。

%%%%%%%%%%%%%%%%%%%%%%%%%%%%%%%%%%%
\vspace{1em}
\noindent \textbf{【重要事項】}

\medskip
振幅 = 実効値$\times \sqrt{2}$

%%%%%%%%%%%%%%%%%%%%%%%%%%%%%%%%%%%
\vspace{1em}
\noindent \textbf{【解説】}

\noindent 日本の交流電圧の実効値は 100 [V] なので、振幅 は 100$\times \sqrt{2}$ [V] となる。
