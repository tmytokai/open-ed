\noindent \textbf{【出題意図】}

\noindent 振幅の符号が変わると時間領域アナログサイン波のグラフがどう変化するか理解しているかどうかを確かめる問題である。

%%%%%%%%%%%%%%%%%%%%%%%%%%%%%%%%%%%
\vspace{1em}
\noindent \textbf{【重要事項】}

\[
f(t) = a \cdot \sin( w \cdot t + \phi )
\]
%
\noindent または
%
\[
f(t) = a \cdot \cos( w \cdot t + \phi )
\]

\medskip
\noindent を時間領域アナログサイン波と呼ぶ

\bigskip
\noindent $a$・・・振幅

\bigskip
\noindent $w$ ・・・角周波数 [rad/秒]

\bigskip
\noindent $\phi$ ・・・初期位相 [rad]

\bigskip
\noindent $t$ ・・・時刻 [秒]

%%%%%%%%%%%%%%%%%%%%%%%%%%%%%%%%%%%
\vspace{1em}
\noindent \textbf{【解説】}

\noindent 振幅の符号が逆になると上下が反転したグラフとなる。
