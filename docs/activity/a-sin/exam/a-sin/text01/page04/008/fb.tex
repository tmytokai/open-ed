\noindent \textbf{【出題意図】}

\noindent 周期と進み・遅れの秒数から時間領域アナログサイン波の初期位相を求めることができるかどうかを確かめる問題である。

%%%%%%%%%%%%%%%%%%%%%%%%%%%%%%%%%%%
\vspace{1em}
\noindent \textbf{【重要事項】}

\bigskip
\noindent $w$ ・・・角周波数 [rad/秒]

\bigskip
\noindent $\phi$ ・・・初期位相 [rad]

\bigskip
$\textrm{T}$・・・周期 [秒]、$\textrm{T} = \frac{1}{f} = \frac{2\pi}{w}$

\begin{center}
\begin{tabularx}{0.9\fbwidth}{|X|X|X|X|}
\hline
$\phi$ の符号&  位相が・・・   & 秒で言い換えると・・・   & 並行移動方向と距離 \\
\hline
プラス &  $\phi$ [rad] 進んでいる & $\phi/w$ [秒] 進んでいる & 左へ$\phi/w$ [秒] \\
\hline
マイナス &  $|\phi|$ [rad] 遅れている & $|\phi|/w$ [秒] 遅れている & 右へ$|\phi|/w$ [秒] \\
\hline
\end{tabularx}

\medskip
%\centering
%\caption{初期位相 $\phi$と進み・遅れの関係 (周期を $\textrm{T}$ を使った場合) }
\begin{tabularx}{0.9\fbwidth}{|X|X|X|X|}
\hline
$\phi$ の符号&  位相が・・・   & 秒で言い換えると・・・   & 並行移動方向と距離 \\
\hline
プラス &  $\phi$ [rad] 進んでいる & $\phi/(2\pi) \cdot \textrm{T}$ [秒] 進んでいる & 左へ $\phi/(2\pi) \cdot \textrm{T}$ [秒] \\
\hline
マイナス &  $|\phi|$ [rad] 遅れている & $|\phi|/(2\pi) \cdot \textrm{T}$ [秒] 遅れている & 右へ $|\phi|/(2\pi) \cdot \textrm{T}$ [秒] \\
\hline
\end{tabularx}
\end{center}

%%%%%%%%%%%%%%%%%%%%%%%%%%%%%%%%%%%
\vspace{1em}
\noindent \textbf{【解説】}

\noindent $\textrm{T} = 1$ [秒]、遅れ $0.25$ 秒より、$|\phi|/(2\pi) \cdot 1 = 0.25$ であるから、
$\phi = -\pi/2$ [rad]が答となる。
