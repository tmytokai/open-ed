\noindent \textbf{【出題意図】}

\noindent 定義式から時間領域アナログサイン波のグラフを求めることができるかどうかを確かめる問題である。

%%%%%%%%%%%%%%%%%%%%%%%%%%%%%%%%%%%
\vspace{1em}
\noindent \textbf{【重要事項】}

\[
f(t) = a \cdot \sin( w \cdot t + \phi )
\]
%
\noindent または
%
\[
f(t) = a \cdot \cos( w \cdot t + \phi )
\]

\medskip
\noindent を時間領域アナログサイン波と呼ぶ

\bigskip
\noindent $a$・・・振幅

\bigskip
\noindent $w$ ・・・角周波数 [rad/秒]

\bigskip
\noindent $\phi$ ・・・初期位相 [rad]

\bigskip
\noindent $t$ ・・・時刻 [秒]

\bigskip
$\textrm{T}$・・・周期 [秒]、$\textrm{T} = \frac{1}{f} = \frac{2\pi}{w}$

%%%%%%%%%%%%%%%%%%%%%%%%%%%%%%%%%%%
\vspace{1em}
\noindent \textbf{【解説】}

\noindent $a = 2$、$w = \pi$、$\phi = 0$ より、振幅が $2$ で周期が $\textrm{T} =2$ [秒]である通常の sin 関数のグラフが答となる。
