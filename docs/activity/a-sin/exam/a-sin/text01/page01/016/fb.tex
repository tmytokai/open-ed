\noindent \textbf{【出題意図】}

\noindent グラフから時間領域アナログサイン波の式を求めることができるかどうかを確かめる問題である。

%%%%%%%%%%%%%%%%%%%%%%%%%%%%%%%%%%%
\vspace{1em}
\noindent \textbf{【重要事項】}

\[
f(t) = a \cdot \sin( w \cdot t + \phi )
\]
%
\noindent または
%
\[
f(t) = a \cdot \cos( w \cdot t + \phi )
\]

\medskip
\noindent を時間領域アナログサイン波と呼ぶ

\bigskip
\noindent $a$・・・振幅

\bigskip
\noindent $w$ ・・・角周波数 [rad/秒]

\bigskip
\noindent $\phi$ ・・・初期位相 [rad]

\bigskip
\noindent $t$ ・・・時刻 [秒]

\bigskip
$\textrm{T}$・・・周期 [秒]、$\textrm{T} = \frac{1}{f} = \frac{2\pi}{w}$

%%%%%%%%%%%%%%%%%%%%%%%%%%%%%%%%%%%
\vspace{1em}
\noindent \textbf{【解説】}

\noindent グラフより振幅が $a = 1$ かつ周期が $\textrm{T} = 1$ [秒]、つまり角周波数が $w = 2\pi$ [rad/秒]の sin 関数が求める式となる。
