\noindent \textbf{【出題意図】}

\noindent 周期性時間領域アナログ信号の基本周波数から周期を求めることができるかどうかを確かめる問題である。

%%%%%%%%%%%%%%%%%%%%%%%%%%%%%%%%%%%
\vspace{1em}
\noindent \textbf{【重要事項】}

\[
f(t) = f(t + n \cdot \textrm{T})
\]

\noindent の関係を満たす時間領域アナログ信号を周期性時間領域アナログ信号という。

\medskip
\noindent $\textrm{T}$ ・・・周期、 実数の\underline{定数}、単位は[秒]、範囲は $\textrm{T} > 0$

\medskip
\noindent $n$ ・・・任意の整数($n = 0,\pm1,\pm2,\cdots$)

\[
f_1 = \frac{1}{\textrm{T}}
\]

\noindent を基本周波数という。単位は [Hz]

\[
w_1 = 2\pi \cdot f_1 = \frac{2\pi}{\textrm{T}}
\]

\noindent を基本角周波数という。単位は [rad/秒] 


%%%%%%%%%%%%%%%%%%%%%%%%%%%%%%%%%%%
\vspace{1em}
\noindent \textbf{【解説】}

\noindent $f_1 = 4$ [Hz] より $\textrm{T}=1/4$ [秒] となる。
