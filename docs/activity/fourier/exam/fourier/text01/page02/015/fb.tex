\noindent \textbf{【出題意図】}

\noindent フーリエの定理について理解しているか確かめる問題である。

%%%%%%%%%%%%%%%%%%%%%%%%%%%%%%%%%%%
\vspace{1em}
\noindent \textbf{【重要事項】}

\noindent フーリエの定理

\smallskip
$f(t)$ が周期 $\textrm{T}$ [秒] の周期性時間領域アナログ信号ならば、$f(t)$ を

\smallskip
1. 直流成分

\smallskip
2. 基本角周波数 $w_1$ [rad/秒]を持つ時間領域アナログサイン波 (基本波と呼ぶ)

\smallskip
3. 基本角周波数 $w_1$ [rad/秒]の正整数倍の角周波数を持つ無限個の時間領域アナログサイン波 (高調波と呼ぶ)

\smallskip
の和に分解できる。


%%%%%%%%%%%%%%%%%%%%%%%%%%%%%%%%%%%
\vspace{1em}
\noindent \textbf{【解説】}

\noindent フーリエの定理は「周期的な信号は異なる周波数のサイン波が複数足し合わされて出来ている」ことを主張する定理である。
