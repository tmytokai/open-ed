\noindent \textbf{【出題意図】}

\bigskip
\noindent 複素フーリエ級数展開の式から複素フーリエ係数 $\textrm{C}[k]$ を求めることができるかどうかを確かめる問題である。

%%%%%%%%%%%%%%%%%%%%%%%%%%%%%%%%%%%
\vspace{1em}
\noindent \textbf{【重要事項】}

\medskip
\noindent ・複素フーリエ級数展開の定義

\begin{align*}
f(t) = \sum_{k = -\infty}^{\infty} 
\left \{
\textrm{C}[k] \cdot \textrm{e}^{\{j \cdot k \cdot w_1 \cdot t \}}
\right \}
\end{align*}

\medskip
\noindent $f(t)$ ・・・周期 $\textrm{T}$ [秒] の周期性時間領域アナログ信号

\medskip
\noindent $w_1$ ・・・ 基本角周波数、$w_1 = 2\pi/\textrm{T}$、単位は [rad/秒]

\medskip
\noindent $\textrm{C}[k]$ ・・・ $k$番目の複素フーリエ係数

\begin{align*}
\textrm{C}[k] = \frac{1}{\textrm{T}} \int_{-\textrm{T}/2}^{\textrm{T}/2} 
\left \{
f(t) \cdot \textrm{e}^{\{-j \cdot k \cdot w_1 \cdot t \}} 
\right \}
\textrm{dt}
\ ,\  (k = 0, \pm 1, \pm 2, \cdots)
\end{align*}

\bigskip
\noindent ・ $\textrm{C}[k]$ と $\textrm{C}[-k]$ は複素共役関係、つまり

\begin{align*}
\textrm{C}[k] = \textrm{C}^{*}[-k]
\end{align*}

\begin{align*}
\textrm{C}^{*}[k] = \textrm{C}[-k]
\end{align*}

\bigskip

%%%%%%%%%%%%%%%%%%%%%%%%%%%%%%%%%%%
\vspace{1em}
\noindent \textbf{【解説】}

\bigskip
\noindent $\textrm{C}[1] = 1 \cdot \textrm{e}^{\{ j \cdot \pi/4 \}}$ より求まる。
