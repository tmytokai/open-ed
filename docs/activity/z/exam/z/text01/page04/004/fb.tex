\noindent \textbf{【出題意図】}

\bigskip
\noindent 有限長の時間領域ディジタル信号の Z 変換を求めることができるかどうかを確かめる問題である。

%%%%%%%%%%%%%%%%%%%%%%%%%%%%%%%%%%%
\vspace{1em}
\noindent \textbf{【重要事項】}

\medskip
正整数 $\textrm{L}$ が有限の時、$f[0]$から$f[\textrm{L}-1]$まで値が入っていて、残りは全て $f[i]=0, (i=\textrm{L},\textrm{L}+1,\cdots)$ である時間領域ディジタル信号の Z 変換は以下の通りである。

\begin{align*}
\textrm{F}(z) 
& = \sum_{i=0}^\infty \{ f[i] \cdot z^{-i} \} \\
& = \sum_{i=0}^{\textrm{L}-1} \{ f[i] \cdot z^{-i} \} \\
& = f[0] + \frac{f[1]}{z} + \frac{f[2]}{z^2} + \cdots + \frac{f[\textrm{L}-1]}{z^{\textrm{L}-1}}
\end{align*}

\medskip
\noindent 収束領域は「原点 $z=0$ を除く Z 平面全域」となる。

\bigskip

%%%%%%%%%%%%%%%%%%%%%%%%%%%%%%%%%%%
\vspace{1em}
\noindent \textbf{【解説】}

\bigskip
\noindent 重要事項で示した $\textrm{F}(z)$ の式から元の $f[i]$ の値が求まる。
