\noindent \textbf{【出題意図】}

\bigskip
\noindent $n \ne 0 $ のときのディジタルインパルス信号の Z 変換を求めることができるかどうかを確かめる問題である。

%%%%%%%%%%%%%%%%%%%%%%%%%%%%%%%%%%%
\vspace{1em}
\noindent \textbf{【重要事項】}

\medskip
$n \ne 0 $ のとき、ディジタルインパルス信号 

\begin{align}
f[i] =
\begin{cases}
1 & (i=n) \\ 
0 & (i \neq n)
\end{cases}
\end{align}

の Z 変換は以下の式で表される。

\begin{align*}
\textrm{F}(z) = \sum_{i=0}^{n} \{ f[i] \cdot z^{-i} \} \\
& = 0 + \frac{0}{z} + \frac{0}{z^2} + \cdots + \frac{1}{z^{n}} \\
& = \frac{1}{z^{n}} (= z^{-n})
\end{align*}

\medskip
\noindent 収束領域は「$z=0$ を除くZ平面全域」となる。

\bigskip

%%%%%%%%%%%%%%%%%%%%%%%%%%%%%%%%%%%
\vspace{1em}
\noindent \textbf{【解説】}

\bigskip
\noindent $n = 1$ なので重要事項より $\textrm{F}(z) = z^{-1} $ が求まる。
