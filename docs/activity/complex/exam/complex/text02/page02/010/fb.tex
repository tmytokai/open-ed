\noindent \textbf{【出題意図】}

\bigskip
\noindent 与えられた時間領域複素正弦波 $z(t)$ の式から周期を求めることができるかどうかを確かめる問題である。

%%%%%%%%%%%%%%%%%%%%%%%%%%%%%%%%%%%
\vspace{1em}
\noindent \textbf{【重要事項】}

\[
z(t) =  \left \{ a \cdot \textrm{e}^{\{j \cdot \phi\}} \right \} 
\cdot \textrm{e}^{\{j \cdot w \cdot t \}}
\]

\bigskip
\noindent を時間領域複素正弦波と呼び、複素平面の上の回転運動体(ベクトル)となる。

\bigskip
\noindent\quad  $a$・・・振幅(または半径)

\bigskip
\noindent\quad $w$ ・・・角周波数、単位は [rad/秒]

\bigskip
\noindent\quad $\phi$ ・・・初期位相、単位は [rad]

\[
w = 2\pi \cdot f
\]

\[
f = \frac{w}{2\pi} 
\]

\[
\textrm{T} = \frac{1}{|f|} = \frac{2\pi}{|w|}   \ \mbox{※ 周期は正なので絶対値を取る}
\]

\bigskip

%%%%%%%%%%%%%%%%%%%%%%%%%%%%%%%%%%%
\vspace{1em}
\noindent \textbf{【解説】}

\bigskip
\noindent $\textrm{T} = \frac{2\pi}{|w|} = 2\pi/(\pi/2) = 4$ により求まる。
