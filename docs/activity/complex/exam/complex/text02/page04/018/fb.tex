\noindent \textbf{【出題意図】}

\bigskip
\noindent サイン波を含む式を変形して時間領域複素正弦波を含む式を求めることができるかどうかを確かめる問題である。

%%%%%%%%%%%%%%%%%%%%%%%%%%%%%%%%%%%
\vspace{1em}
\noindent \textbf{【重要事項】}

\medskip
\noindent ・ 時間領域アナログサイン波はオイラー公式を用いて2つの時間領域複素正弦波の和に分解できる

\medskip
\noindent \mbox{(sin 版)}

\[
a \cdot \sin ( w \cdot t + \phi )
=  \left \{ \frac{a}{2} \cdot \textrm{e}^{\{-j (\phi-\pi/2) \}} \right \} \cdot \textrm{e}^{\{-j \cdot w \cdot t \}}
+  \left \{ \frac{a}{2} \cdot \textrm{e}^{\{j (\phi-\pi/2) \}} \right \} \cdot \textrm{e}^{\{j \cdot w \cdot t \}}
\]

\medskip
\noindent \mbox{(cos 版)}

\[
a \cdot \cos ( w \cdot t + \phi )
=  \left \{ \frac{a}{2} \cdot \textrm{e}^{\{-j \cdot \phi \}} \right \} \cdot \textrm{e}^{\{-j \cdot w \cdot t \}}
+  \left \{ \frac{a}{2} \cdot \textrm{e}^{\{j  \cdot \phi \}} \right \} \cdot \textrm{e}^{\{j \cdot w \cdot t \}}
\]

\bigskip

%%%%%%%%%%%%%%%%%%%%%%%%%%%%%%%%%%%
\vspace{1em}
\noindent \textbf{【解説】}

\bigskip

\begin{align*}
& 2 \cdot \sin(w/2 \cdot t) \cdot \cos( w/2 \cdot t) \\
& = 
2 \cdot
\frac{1}{2}\cdot\left \{
  \textrm{e}^{\{ j \cdot \pi/2 \}} \cdot \textrm{e}^{\{-j \cdot w/2 \cdot t \}} 
+ \textrm{e}^{\{-j \cdot \pi/2 \}} \cdot \textrm{e}^{\{ j \cdot w/2 \cdot t \}} 
\right \}
\cdot
\frac{1}{2}\cdot\left \{
  \textrm{e}^{\{-j \cdot w/2 \cdot t \}} 
+ \textrm{e}^{\{ j \cdot w/2 \cdot t \}} 
\right \} \\
& =
  \frac{1}{2} \cdot \textrm{e}^{\{ j \cdot \pi/2 \}} \cdot \textrm{e}^{\{-j \cdot w \cdot t \}}
+ \frac{1}{2} \cdot \textrm{e}^{\{-j \cdot \pi/2 \}} \cdot \textrm{e}^{\{ j \cdot w \cdot t \}}
\end{align*}
