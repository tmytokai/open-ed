\noindent \textbf{【出題意図】}

\bigskip
\noindent サイン波を含む式を変形して時間領域複素正弦波を含む式を求めることができるかどうかを確かめる問題である。

%%%%%%%%%%%%%%%%%%%%%%%%%%%%%%%%%%%
\vspace{1em}
\noindent \textbf{【重要事項】}

\medskip
\noindent 時間領域アナログサイン波同士の計算はオイラー公式を用いて2つの時間領域複素正弦波の和に分解することで計算や証明が楽になることが多い。

\bigskip
%%%%%%%%%%%%%%%%%%%%%%%%%%%%%%%%%%%
\vspace{1em}
\noindent \textbf{【解説】}

\bigskip

\begin{align*}
\{2 \cdot \sin (2\pi t) \}^2
&=
\left [
  \frac{2}{2}\cdot\textrm{e}^{\{j \cdot \pi/2 \}}\cdot\textrm{e}^{\{-j \cdot 2\pi \cdot t \}} 
+ \frac{2}{2}\cdot\textrm{e}^{\{-j \cdot \pi/2 \}}\cdot\textrm{e}^{\{j \cdot 2\pi \cdot t \}} 
\right ]^2 \\
&=
  \left [ \textrm{e}^{\{j \cdot \pi/2 \}}\cdot\textrm{e}^{\{-j \cdot 2\pi \cdot t \}} \right ]^2 \\
&+ 2 \cdot \left [ \textrm{e}^{\{j \cdot \pi/2 \}}\cdot\textrm{e}^{\{-j \cdot 2\pi \cdot t \}} \right ]
     \cdot \left [ \textrm{e}^{\{-j \cdot \pi/2 \}}\cdot\textrm{e}^{\{j \cdot 2\pi \cdot t \}} \right ] \\
&+ \left [ \textrm{e}^{\{-j \cdot \pi/2 \}}\cdot\textrm{e}^{\{j \cdot 2\pi \cdot t \}} \right ]^2 \\
&=
  \textrm{e}^{\{ j \cdot \pi \}} \cdot\ \textrm{e}^{\{-j \cdot 4\pi \cdot t \}}
+2 \cdot \textrm{e}^{0}
+ \textrm{e}^{\{-j \cdot \pi \}} \cdot\ \textrm{e}^{\{ j \cdot 4\pi \cdot t \}} \\
&=
-\textrm{e}^{\{-j \cdot 4\pi \cdot t \}}
-\textrm{e}^{\{ j \cdot 4\pi \cdot t \}}
+2
\end{align*}

\smallskip
(別解) 半角公式を使う

\begin{align*}
\{2 \cdot \sin (2\pi t) \}^2 
= 2 \cdot \left \{ 1-\cos( 4 \pi t) \right  \}
= 2 -\textrm{e}^{\{-j \cdot 4\pi \cdot t \}} -\textrm{e}^{\{ j \cdot 4\pi \cdot t \}}
\end{align*}
