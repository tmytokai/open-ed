\noindent \textbf{【出題意図】}

\bigskip
\noindent 時間領域複素正弦波の自然対数の動き方を求めることができるかどうかを確かめる問題である。

%%%%%%%%%%%%%%%%%%%%%%%%%%%%%%%%%%%
\vspace{1em}
\noindent \textbf{【重要事項】}

\medskip
\noindent 自然対数の公式より、時間領域複素正弦波 

\[
z(t) =  \left \{ a \cdot \textrm{e}^{\{j \cdot \phi\}} \right \} \cdot \textrm{e}^{\{j \cdot w \cdot t \}}
\]

\bigskip
\noindent の自然対数は $\log_e z(t) =  \log_e a + j \cdot \phi + j \cdot w \cdot t$ と求められる。

\bigskip

%%%%%%%%%%%%%%%%%%%%%%%%%%%%%%%%%%%
\vspace{1em}
\noindent \textbf{【解説】}

\bigskip
\noindent グラフより $\log_e z(t) =  2 + j \cdot t = \log_e \textrm{e}^2 + j \cdot 0 + j \cdot 1 \cdot t$ であるので、$a = \textrm{e}^2$、$\phi = 0$、$w=1$ より求まる。
