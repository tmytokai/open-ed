\noindent \textbf{【出題意図】}

\bigskip
\noindent サイン波を含む式を変形して時間領域複素正弦波を含む式を求めることができるかどうかを確かめる問題である。

%%%%%%%%%%%%%%%%%%%%%%%%%%%%%%%%%%%
\vspace{1em}
\noindent \textbf{【重要事項】}

\medskip
\noindent ・ 時間領域アナログサイン波はオイラー公式を用いて2つの時間領域複素正弦波の和に分解できる

\medskip
\noindent \mbox{(sin 版)}

\[
a \cdot \sin ( w \cdot t + \phi )
=  \left \{ \frac{a}{2} \cdot \textrm{e}^{\{-j (\phi-\pi/2) \}} \right \} \cdot \textrm{e}^{\{-j \cdot w \cdot t \}}
+  \left \{ \frac{a}{2} \cdot \textrm{e}^{\{j (\phi-\pi/2) \}} \right \} \cdot \textrm{e}^{\{j \cdot w \cdot t \}}
\]

\medskip
\noindent \mbox{(cos 版)}

\[
a \cdot \cos ( w \cdot t + \phi )
=  \left \{ \frac{a}{2} \cdot \textrm{e}^{\{-j \cdot \phi \}} \right \} \cdot \textrm{e}^{\{-j \cdot w \cdot t \}}
+  \left \{ \frac{a}{2} \cdot \textrm{e}^{\{j  \cdot \phi \}} \right \} \cdot \textrm{e}^{\{j \cdot w \cdot t \}}
\]

\bigskip
\noindent  $a$・・・振幅

\bigskip
\noindent $w$ ・・・角周波数、単位は [rad/秒]

\bigskip
\noindent $\phi$ ・・・初期位相、単位は [rad]

\bigskip

%%%%%%%%%%%%%%%%%%%%%%%%%%%%%%%%%%%
\vspace{1em}
\noindent \textbf{【解説】}

\bigskip

\[
\{2 \cos (wt) \}^2 = 
\left [ \textrm{e}^{\{-j \cdot w \cdot t \}} 
+ \textrm{e}^{\{j \cdot w \cdot t \}} 
\right ]^2
=
\textrm{e}^{\{-j \cdot 2w \cdot t \}}
+\textrm{e}^{\{ j \cdot 2w \cdot t \}}
+2
\]
