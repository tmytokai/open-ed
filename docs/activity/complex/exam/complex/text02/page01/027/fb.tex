\noindent \textbf{【出題意図】}

\bigskip
\noindent 与えられた時間領域複素信号 $z(t)$ のある時刻における位置を求めることができるかどうかを確かめる問題である。

%%%%%%%%%%%%%%%%%%%%%%%%%%%%%%%%%%%
\vspace{1em}
\noindent \textbf{【重要事項】}

\bigskip
\noindent ・ 絶対値及び偏角が $t$ [秒] の関数で表される複素数

\[
z(t) = |z(t)| \cdot \textrm{e}^{\{j \cdot \angle \ z(t)\}}
\]

\medskip
\noindent を「時間領域(アナログ)複素信号」と呼び、複素平面の上を移動する運動体(ベクトル)となる。

%%%%%%%%%%%%%%%%%%%%%%%%%%%%%%%%%%%
\vspace{1em}
\noindent \textbf{【解説】}

\bigskip
\noindent 直交形式で表された複素信号である。そのまま時刻を代入して $z(\pi/4) = \sqrt{2} \cdot \exp(j \cdot \pi/4) = 1 + j $ により位置を求められる。
