\noindent \textbf{【出題意図】}

\bigskip
\noindent 与えられた時間領域複素信号 $z(t)$ の動き方を求めることができるかどうかを確かめる問題である。

%%%%%%%%%%%%%%%%%%%%%%%%%%%%%%%%%%%
\vspace{1em}
\noindent \textbf{【重要事項】}

\bigskip
\noindent ・ 絶対値及び偏角が $t$ [秒] の関数で表される複素数

\[
z(t) = |z(t)| \cdot \textrm{e}^{\{j \cdot \angle \ z(t)\}}
\]

\medskip
\noindent を「時間領域(アナログ)複素信号」と呼び、複素平面の上を移動する運動体(ベクトル)となる。

%%%%%%%%%%%%%%%%%%%%%%%%%%%%%%%%%%%
\vspace{1em}
\noindent \textbf{【解説】}

\bigskip
\noindent 実際に具体的な値をいくつか選んで代入してみることで動き方を求められる。
