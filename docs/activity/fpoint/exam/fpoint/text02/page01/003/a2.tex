\begin{center}
12.5
\end{center}
