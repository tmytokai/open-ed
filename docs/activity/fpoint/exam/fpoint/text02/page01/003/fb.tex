\noindent \textbf{【出題意図】}

\noindent 10 進数の小数を固定小数点数形式の 2 進数または 16 進数と相互変換することができるかどうかを確かめる問題である。

%%%%%%%%%%%%%%%%%%%%%%%%%%%%%%%%%%%
\vspace{1em}
\noindent \textbf{【重要事項】}

\noindent 変換方法

\medskip
\noindent ・10進数 → 2 進数

\medskip
\noindent ステップ1. 整数部と小数部で別々に 2 進数を求める 

\noindent ステップ2. 整数部の 2 進数と小数部の 2 進数を足し合わせると固定小数点形式の 2 進数になる 

\medskip
\noindent ・2 進数 → 16 進数

\medskip
\noindent ステップ1. 整数部と小数部で別々に 4 bit ごとに区切る。4bitに満たない場合は0で埋める

\noindent ステップ2. 0〜F までの16進数を割り当てる


\medskip
\noindent ・2、16 進数 → 10 進数

\medskip
\noindent ステップ1. 16 進数の場合は一度 2 進数に戻す

\noindent ステップ2. 整数部と小数部で別々に 10 進数を求める 

\noindent ステップ3. 整数部の 10 進数と小数部の 10 進数を足し合わせると求める 10 進数になる 


%%%%%%%%%%%%%%%%%%%%%%%%%%%%%%%%%%%
\vspace{1em}
\noindent \textbf{【解説】}

\noindent 整数部が 0b1100 → 12、小数部が 0.0001 → 0.0625 となることより求まる。
