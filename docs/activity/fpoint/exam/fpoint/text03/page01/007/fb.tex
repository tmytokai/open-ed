\noindent \textbf{【出題意図】}

\noindent 10 進数の 0 以上の小数を浮動小数点数形式(IEEE754、単精度)の 2、16 進数と相互変換できるかどうかを確かめる問題である。

%%%%%%%%%%%%%%%%%%%%%%%%%%%%%%%%%%%
\vspace{1em}
\noindent \textbf{【重要事項】}

\noindent 変換方法

\medskip
\noindent 2、16 進数(単精度) → 10 進数

\noindent ステップ1.  16 進数の場合は一度 2 進数に戻す

\noindent ステップ2.  2 進数の先頭ビットを「符号部」とする

\noindent ステップ3.  2 進数の1〜8ビットを「指数部」 として取り出し、 E = 指数部の 10 進数 - 127 を計算する。

\noindent ステップ4.  2 進数の残りの 23 ビットを「仮数部」として取り出し、仮数部の先頭に「1.」を追加する。小数点を E だけ移動(右方向がプラス)して固定小数点数形式に変換する。ステップ5.  ステップ 4 で求めた固定小数点数形式の 2 進数を 10 進数に変換する

\noindent ステップ4.  符号部が 1 なら先頭にマイナス符号を付ける 


%%%%%%%%%%%%%%%%%%%%%%%%%%%%%%%%%%%
\vspace{1em}
\noindent \textbf{【解説】}

\noindent 符号部=0b 0、指数部=0b 10000101(=10進133)、仮数部=0b 00001010001 (以下0) より、固定小数点数にすると  +0b 1000010.10001 なので、10進数は 66.53125
