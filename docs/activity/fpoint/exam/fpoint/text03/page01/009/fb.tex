\noindent \textbf{【出題意図】}

\noindent 10 進数の 0 以上の小数を浮動小数点数形式(IEEE754、単精度)の 2、16 進数と相互変換できるかどうかを確かめる問題である。

%%%%%%%%%%%%%%%%%%%%%%%%%%%%%%%%%%%
\vspace{1em}
\noindent \textbf{【重要事項】}

\noindent 変換方法

\medskip
\noindent ・10進数 → 2 進数

\medskip
\noindent ステップ1. 「符号部」を求める。10 進数の小数が 0 以上なら 0、 負なら 1 

\noindent ステップ2.  10 進数の小数の「絶対値」の固定小数点数形式の 2 進数を求める 

\noindent ステップ3.  ステップ 2 で求めた 2 進数の小数点を左右に移動して整数部を 1 桁の 1 だけにする。移動した回数を E とする(右方向がマイナス)。小数部分を M とする。

\noindent ステップ4.  「指数部」(単精度の場合は 8 ビット)を求める。単精度の場合は、ステップ 3 で求めた E に 127 を足して 8 ビットの 2 進数に変換する

\noindent ステップ5. 「 仮数部」(単精度の場合は 23 ビット)を求める。単精度の場合は、ステップ 3 で求めた M が 23 ビットよりも長い場合は 24 ビット目が 0 なら 24 ビット目を切り捨て、1 なら 24 ビット目を切り上げる(※)

\noindent 23 ビットより短い場合は後ろを 0 で埋める ※この方式の事を「最近接偶数丸め」言う。単純に24bit目以降を切り捨てる場合もある 

\noindent ステップ6.  符号部、指数部、仮数部を繋げ、単精度の場合は 32 bit の 2 進数にする


%%%%%%%%%%%%%%%%%%%%%%%%%%%%%%%%%%%
\vspace{1em}
\noindent \textbf{【解説】}

\noindent 重要事項より求まる。
