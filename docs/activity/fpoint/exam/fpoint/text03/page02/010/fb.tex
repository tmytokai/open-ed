\noindent \textbf{【出題意図】}

\noindent IEEE754形式における特殊な値を求めることができるかどうかを確かめる問題である。

%%%%%%%%%%%%%%%%%%%%%%%%%%%%%%%%%%%
\vspace{1em}
\noindent \textbf{【重要事項】}

\begin{center}
\begin{tabularx}{0.9\fbwidth}{|X|X|X|X|}
\hline
値 & 符号部 & 指数部 & 仮数部 \\
\hline
+0 & 0 & 全て0 & 全て 0 \\
\hline
-0 & 1 & 全て0 & 全て 0 \\
\hline
+Infinity & 0 & 全て1 & 全て 0 \\
\hline
-Infinity & 1 & 全て1 & 全て 0 \\
\hline
NaN & 0または1 & 全て1 & 0以外 \\
\hline
\end{tabularx}
\end{center}


%%%%%%%%%%%%%%%%%%%%%%%%%%%%%%%%%%%
\vspace{1em}
\noindent \textbf{【解説】}

\noindent 0x FFC = 0b 1 11111111 100 より求まる。
