\noindent \textbf{【出題意図】}

\noindent 浮動小数点数形式で生じる誤差について理解しているかどうかを確かめる問題である。

%%%%%%%%%%%%%%%%%%%%%%%%%%%%%%%%%%%
\vspace{1em}
\noindent \textbf{【重要事項】}

\medskip
\noindent ・丸め誤差

\medskip
\noindent 小数を含む実数を固定小数点数形式や浮動小数点数形式で 2 進数化すると生じる誤差。

\noindent 精度を上げたり、計算方法を工夫することである程度防ぐことが出来る。

\medskip
\noindent ・桁落ち

\medskip
\noindent 近い値の小数同士で引き算をすると仮数部の有効桁数が減ることで生じる誤差。

\noindent 精度を上げたり、計算方法を工夫することである程度防ぐことが出来る。


%%%%%%%%%%%%%%%%%%%%%%%%%%%%%%%%%%%
\vspace{1em}
\noindent \textbf{【解説】}

\noindent 0.261 をIEEE754(単精度)形式に変換した後10進数に戻すと 0.26100000739097595 になり丸め誤差が生じている。
