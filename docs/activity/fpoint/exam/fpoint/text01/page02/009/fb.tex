\noindent \textbf{【出題意図】}

\noindent 10 進数の負を含む整数を 2 の補数を使って 2、16 進数と相互変換することができるかどうかを確かめる問題である。

%%%%%%%%%%%%%%%%%%%%%%%%%%%%%%%%%%%
\vspace{1em}
\noindent \textbf{【重要事項】}

\noindent 変換方法

\medskip
\noindent ・10進数 → 2 進数

\noindent ステップ1.  0 以上の整数の場合はそのまま 2 進数を求めて終わり

\noindent ステップ2.  マイナス符号を取り除いた絶対値の 2 進数を求める 

\noindent ステップ3.  長さが指定されたビット数未満の場合は先頭に 0 を追加して指定されたビット数にする 

\noindent ステップ4.  各桁の 0 と 1 をひっくり返す 

\noindent ステップ5.  1 を足す 


\medskip
\noindent ・2 進数 → 16 進数

\noindent ステップ1. 4  ビットごとに区切る。4 ビットに満たない所は 0 で埋める

\noindent ステップ2.  0〜F までの 16 進数を割り当てる

\medskip
\noindent ・2、 16 進数 → 10 進数

\noindent ステップ1.  16 進数の場合は一度 2 進数に戻す

\noindent ステップ2.  先頭ビットが 0 の場合は 0 以上の値なのでそのまま 10 進数を求めて終わり

\noindent ステップ3.  1 を引く

\noindent ステップ4.  各桁の 0 と 1 をひっくり返す

\noindent ステップ5.  10 進数を求める

\noindent ステップ6.  先頭にマイナス符号を付ける


%%%%%%%%%%%%%%%%%%%%%%%%%%%%%%%%%%%
\vspace{1em}
\noindent \textbf{【解説】}

\noindent 重要事項より求まる。
