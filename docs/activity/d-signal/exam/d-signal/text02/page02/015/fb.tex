
\noindent \textbf{【出題意図】}

\noindent ナイキスト周波数からサンプリング角周波数 $w_s$ を求めることができるかどうかを確かめる問題である。

%%%%%%%%%%%%%%%%%%%%%%%%%%%%%%%%%%%
\vspace{1em}
\noindent \textbf{【重要事項】}

\medskip
\noindent ・ $f_s/2$ の事を「ナイキスト周波数」、$2\pi \cdot f_s/2$ の事を「ナイキスト角周波数」と呼ぶ

\medskip
\noindent ・ 元の時間領域アナログ信号 $f(t)$ にナイキスト周波数以上の周波数のアナログサイン波が含まれていると正しくサンプリング出来ず変な波形になる

\medskip
\noindent ・ ナイキスト周波数以上の周波数のアナログサイン波をサンプリングすると、そのサイン波は「エイリアシング」または「折り返しひずみ」と呼ばれるノイズに変わる

%%%%%%%%%%%%%%%%%%%%%%%%%%%%%%%%%%%
\vspace{1em}
\noindent \textbf{【解説】}

\noindent  $w_s = 2 \pi \times 2 \times$ ナイキスト周波数 [Hz] の関係より求まる。
