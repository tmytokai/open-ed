\noindent \textbf{【出題意図】}

\noindent サンプリング周波数に関する知識を確かめる問題である。

%%%%%%%%%%%%%%%%%%%%%%%%%%%%%%%%%%%
\vspace{1em}
\noindent \textbf{【重要事項】}

\medskip
\noindent ・ 「サンプリング」はアナログ信号 $f(t)$ の値をある間隔ごとに飛び飛びに取得していく処理

\medskip
\noindent ・ サンプリング周波数 $f_s$ ・・・ アナログ信号 $f(t)$ に対して $1$ 秒間に何回サンプリングするかを表す正の整数値、単位は [Hz](ヘルツ)

\medskip
\noindent ・ サンプリング角周波数 $w_s$ ・・・ $w_s = 2\pi\cdot f_s$ と $f_s$ を角周波数に変換した値、単位は [rad/秒](ラジアン毎秒)

\medskip
\noindent ・ サンプリング間隔 $\tau$ (タウ) ・・・ $f(t)$ に対して何秒おきにサンプリングするかを表す正の整数値、単位は [秒]

\medskip
\noindent ・ $\tau = 1/f_s$ の関係がある

\medskip
\noindent ・ $f_s/2$ の事を「ナイキスト周波数」、$2\pi \cdot f_s/2$ の事を「ナイキスト角周波数」と呼ぶ

\medskip
\noindent ・ 元の時間領域アナログ信号 $f(t)$ にナイキスト周波数以上の周波数のアナログサイン波が含まれていると正しくサンプリング出来ず変な波形になる

\medskip
\noindent ・ ナイキスト周波数以上の周波数のアナログサイン波をサンプリングすると、そのサイン波は「エイリアシング」または「折り返しひずみ」と呼ばれるノイズに変わる

%%%%%%%%%%%%%%%%%%%%%%%%%%%%%%%%%%%
\vspace{1em}
\noindent \textbf{【解説】}

\noindent $f_s$ [Hz] を高くするとナイキスト周波数が高くなるので折り返しひずみが出にくくなる。
