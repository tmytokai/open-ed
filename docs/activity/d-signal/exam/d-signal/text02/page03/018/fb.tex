\noindent \textbf{【出題意図】}

\noindent エイリアシング(折り返しひずみ)が起こる条件を理解しているかどうかを確かめる問題である。

%%%%%%%%%%%%%%%%%%%%%%%%%%%%%%%%%%%
\vspace{1em}
\noindent \textbf{【重要事項】}

\medskip
\noindent ・ $f_s/2$ の事を「ナイキスト周波数」、$2\pi \cdot f_s/2$ の事を「ナイキスト角周波数」と呼ぶ

\medskip
\noindent ・ 元の時間領域アナログ信号 $f(t)$ にナイキスト周波数以上の周波数のアナログサイン波が含まれていると正しくサンプリング出来ず変な波形になる

\medskip
\noindent ・ ナイキスト周波数以上の周波数のアナログサイン波をサンプリングすると、そのサイン波は「エイリアシング」または「折り返しひずみ」と呼ばれるノイズに変わる

%%%%%%%%%%%%%%%%%%%%%%%%%%%%%%%%%%%
\vspace{1em}
\noindent \textbf{【解説】}

\noindent  元の信号に含まれるアナログサイン波の最大周波数がナイキスト周波数 $f_s/2$ より小さければエイリアシングは生じない。
