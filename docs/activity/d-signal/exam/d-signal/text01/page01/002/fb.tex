\noindent \textbf{【出題意図】}

\noindent ディジタル信号を扱う分野を理解しているかどうかを確かめる問題である。

%%%%%%%%%%%%%%%%%%%%%%%%%%%%%%%%%%%
\vspace{1em}
\noindent \textbf{【重要事項】}

\medskip
\noindent ・ 関数の出力値 $f[i]$ が実数値又は複素数で、かつ独立変数 $i$ が飛び飛びの離散値を取るとき、 この関数 $f[i]$ のことをディジタル信号、又はディジタル信号列という。離散値とは $i=0,\ 1,\ 2$のような整数値又は$i=0.10,\ 0.11,\ 0.12$ のような飛び飛びの実数値のことである。

%%%%%%%%%%%%%%%%%%%%%%%%%%%%%%%%%%%
\vspace{1em}
\noindent \textbf{【解説】}

\noindent フラッシュメモリには画像や音声がディジタルデータとして保存されている。
