\noindent \textbf{【出題意図】}

\noindent 定義式から時間領域ディジタル信号 $f[i]$ のグラフを求めることができるかどうかを確かめる問題である。

%%%%%%%%%%%%%%%%%%%%%%%%%%%%%%%%%%%
\vspace{1em}
\noindent \textbf{【重要事項】}

\bigskip
\noindent ・ 時間によって値が変化するディジタル信号 $f[i]$ のことを時間領域ディジタル信号という

\bigskip
\noindent ・ 関数 $f[i]$の値は実数値又は複素数

\bigskip
\noindent ・ 独立変数 $i$ の値は飛び飛びの離散値

\bigskip
\noindent ・ $i$ に単位はないので単に「時刻 $i$」とだけ呼ぶ


%%%%%%%%%%%%%%%%%%%%%%%%%%%%%%%%%%%
\vspace{1em}
\noindent \textbf{【解説】}

\noindent 時刻 $i$ は飛び飛びの離散値を取るので直線や曲線のグラフは間違いである。
あとは $f[i]$ に実際に $i$ の値を代入して確かめることで正しいグラフを求められる。
