\noindent \textbf{【出題意図】}

\noindent 直線のグラフを求めることができるかどうかを確かめる問題である。

%%%%%%%%%%%%%%%%%%%%%%%%%%%%%%%%%%%
\vspace{1em}
\noindent \textbf{【重要事項】}

\bigskip
\noindent ・ 時間によって値が変化するディジタル信号 $f[i]$ のことを時間領域ディジタル信号という

\bigskip
\noindent ・ 関数 $f[i]$の値は実数値又は複素数

\bigskip
\noindent ・ 独立変数 $i$ の値は飛び飛びの離散値

\bigskip
\noindent ・ $i$ に単位はないので単に「時刻 $i$」とだけ呼ぶ


%%%%%%%%%%%%%%%%%%%%%%%%%%%%%%%%%%%
\vspace{1em}
\noindent \textbf{【解説】}

\noindent ディジタル信号ではないので点グラフは除かれる。
また横方向に直線や曲線が描いてあるグラフも除かれる。
あとは対応する時刻から垂直に線を引いてあるグラフを選択する。
