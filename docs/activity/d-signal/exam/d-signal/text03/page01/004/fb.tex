\noindent \textbf{【出題意図】}

\noindent 量子化誤差を求めることができるかどうかを確かめる問題である。

%%%%%%%%%%%%%%%%%%%%%%%%%%%%%%%%%%%
\vspace{1em}
\noindent \textbf{【重要事項】}

\medskip
\noindent ・ 量子化・・ $f[i]$ の出力値(連続的な実数値)をディジタル化(離散化)する処理

\medskip
\noindent ・ 量子化幅 $\Delta$ ・・・ どの間隔で出力値のディジタル化を行うかを決めるパラメータ。単位は扱う信号の種類による(ボルトとかアンペアとか度とかetc.)

\medskip
\noindent ・ 線形量子化 ・・・ 量子化幅  $\Delta$ が可変でなく常に一定である量子化のこと

\medskip
\noindent ・ 非線形量子化 ・・・ 量子化幅  $\Delta$ が可変で状況によって変化する量子化のこと

\medskip
\noindent ・ 量子化誤差 ・・・ 元の $f[i]$ の出力値 $a$ と量子化後の $f[i]$ の出力値 $b$ の差 $a-b$ のこと

%%%%%%%%%%%%%%%%%%%%%%%%%%%%%%%%%%%
\vspace{1em}
\noindent \textbf{【解説】}

\noindent $a = 2.2$、$b=2.0$ より $a-b =0.2$ と答が求まる。
