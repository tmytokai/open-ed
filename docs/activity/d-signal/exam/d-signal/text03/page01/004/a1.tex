\begin{center}
$2.2$
\end{center}
