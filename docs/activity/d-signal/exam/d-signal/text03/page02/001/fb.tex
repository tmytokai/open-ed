\noindent \textbf{【出題意図】}

\noindent 量子化ビット数から値域の分割数を求めることができるかどうかを確かめる問題である。

%%%%%%%%%%%%%%%%%%%%%%%%%%%%%%%%%%%
\vspace{1em}
\noindent \textbf{【重要事項】}

\medskip
\noindent ・ 量子化・・ $f[i]$ の出力値(連続的な実数値)をディジタル化(離散化)する処理

\medskip
\noindent ・ 量子化幅 $\Delta$ ・・・ どの間隔で出力値のディジタル化を行うかを決めるパラメータ。単位は扱う信号の種類による(ボルトとかアンペアとか度とかetc.)

\medskip
\noindent ・ 線形量子化 ・・・ 量子化幅  $\Delta$ が可変でなく常に一定である量子化のこと

\medskip
\noindent ・ 非線形量子化 ・・・ 量子化幅  $\Delta$ が可変で状況によって変化する量子化のこと

\medskip
\noindent ・ 量子化誤差 ・・・ 元の $f[i]$ の出力値と量子化後の $f[i]$ の出力値の差

\medskip
\noindent ・ 量子化ビット数 $q$ [bit] ・・・ 量子化した後のディジタルデータを何 bit で記録するかを表す数字

\medskip
\noindent ・ 線形量子化の場合は$f[i]$の値域を均等に $2^q-1$ 分割することを意味する

\medskip
\noindent ・ 非線形量子化の場合は$f[i]$の値域が均等に分割されるとは限らない

%%%%%%%%%%%%%%%%%%%%%%%%%%%%%%%%%%%
\vspace{1em}
\noindent \textbf{【解説】}

\noindent 量子化ビット数が $q$ [bit]で線形量子化する時、値域が均等に $2^q-1$ 分割されることから求まる。

