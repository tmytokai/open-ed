\noindent \textbf{【出題意図】}

\noindent 量子化されていない時間領域ディジタル信号から量子化した信号を求めることができるかどうかを確かめる問題である。

%%%%%%%%%%%%%%%%%%%%%%%%%%%%%%%%%%%
\vspace{1em}
\noindent \textbf{【重要事項】}

\medskip
\noindent ・ 量子化・・ $f[i]$ の出力値(連続的な実数値)をディジタル化(離散化)する処理

\medskip
\noindent ・ 量子化幅 $\Delta$ ・・・ どの間隔で出力値のディジタル化を行うかを決めるパラメータ。単位は扱う信号の種類による(ボルトとかアンペアとか度とかetc.)

\medskip
\noindent ・ 線形量子化 ・・・ 量子化幅  $\Delta$ が可変でなく常に一定である量子化のこと

\medskip
\noindent ・ 非線形量子化 ・・・ 量子化幅  $\Delta$ が可変で状況によって変化する量子化のこと

\medskip
\noindent ・ 量子化誤差 ・・・ 元の $f[i]$ の出力値と量子化後の $f[i]$ の出力値の差

%%%%%%%%%%%%%%%%%%%%%%%%%%%%%%%%%%%
\vspace{1em}
\noindent \textbf{【解説】}

\noindent $f[i] = \{0.9,\ 1.0,\ 0.3,\ 0.4 \}$ を与えられた条件で量子化すると $f'[i] = \{1.0,\ 1.0,\ 0.0,\ 0.0 \}$ となる。あとは$f'[i]$を二進数のディジタルデータに変換する。
