\correct
\begin{center}
\[
3 \cdot \sin ( \pi/4 \cdot t + \pi/3 )
\]
\end{center}
