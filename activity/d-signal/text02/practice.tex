\documentclass[a4paper]{jarticle}

\def\texdir{../../../tex}
\makeatletter 

\pagestyle{plain}

\parindent = 1zw
\hoffset = -1in
\voffset = -1in
\topmargin = 15mm
\headheight = 0mm
\headsep = 0mm
\oddsidemargin = 15mm
\evensidemargin = 15mm
\textwidth = 210mm  
\textheight = 287mm
\addtolength{\textwidth}{-\oddsidemargin}
\addtolength{\textwidth}{-\evensidemargin}
\addtolength{\textheight}{-\topmargin}
\addtolength{\textheight}{-\headheight}
\addtolength{\textheight}{-\headsep}
\addtolength{\textheight}{-\footskip}

%% enumerate はカッコ付き数字
\renewcommand{\labelenumi}{(\arabic{enumi})}

%%%% 演習ページ用マクロ %%%%

% ページ番号
\newcounter{c-page}
\setcounter{c-page}{0}

% 課題番号
\newcounter{c-prac}
\setcounter{c-prac}{0}

% 合計時間
\newcounter{practime}
\setcounter{practime}{0}

\newcommand{\prachead}[2]{%
\noindent アクティビティ: #1\par%
\noindent 学習項目: #2\par%
\noindent 演習時間: \ref{label:practime} 分\par%
\vspace{2zh}%
\noindent \textbf{\Large 演習}%
}

\newcommand{\pracpage}[1]{%
\stepcounter{c-page}%
\vspace{4zh}%
\noindent%
\textbf{\noindent \textbf{\Large \arabic{c-page}. #1}}%
}

\newcommand{\practeam}[1]{%
\stepcounter{c-prac}%
\vspace{2zh}%
\noindent%
\textbf{演習 \arabic{c-prac} (チーム, #1 分):}%
\advance \c@practime by #1%
}

\newcommand{\pracnote}[1]{%
\stepcounter{c-prac}%
\vspace{2zh}%
\noindent%
\textbf{演習 \arabic{c-prac} (個人ノート, #1 分):}%
\advance \c@practime by #1%
}

\newcommand{\pracpc}[1]{%
\stepcounter{c-prac}%
\vspace{2zh}%
\noindent%
\textbf{演習 \arabic{c-prac} (個人PC, #1 分):}%
\advance \c@practime by #1%
}

\newcommand{\savepractime}{%
\addtocounter{practime}{-1}%
\refstepcounter{practime}%
\label{label:practime}%
}

\makeatother 


\begin{document}

\prachead{ディジタル信号処理の基礎}{[2] サンプリング(標本化)}

%%%%%%%%%%%%%%%%%%%%%%%%
\pracpage{サンプリング(標本化)}

\practeam{10} CD、ハイレゾ、DSD(Direct Stream Digital)、VoLTE(AMR-WB) のサンプリング周波数 $f_s$ [Hz]を調べましょう。\par
\noindent 役割を分担して各自で調べ、結果を出しあってホワイトボードにまとめて下さい。

\practeam{10} サンプリング周波数 $f_s$ [Hz]を大きくした時のメリットとデメリットを話し合ってホワイトボードにまとめて下さい。\par
\noindent ※ 小さくしたときはメリットとデメリットが逆になります。

\pracnote{5} ある時間領域アナログ信号 $f(t)$ を $f_s = 10$ [Hz] でサンプリングした時のサンプリング間隔 $\tau$ [秒] はいくらになるかノート上で計算して下さい。

\pracnote{5} 同様にサンプリング角周波数 $w_s$ [rad/秒] はいくらになるかノート上で計算して下さい。

\pracpc{10} 表計算ソフト使って、時間領域アナログ信号
$f(t) = \sin( 2\pi\cdot 2 \cdot t ) + \sin( 2\pi\cdot 4 \cdot t )$ のグラフを描いてみましょう。\par
\noindent 時刻の範囲は $0 \leq t \leq 4$ [秒] とし、代表点は 0.01 秒刻みとします。 \par
\noindent グラフは点なし、線あり、平滑化なしとします。

\pracpc{10} 表計算ソフト使って、演習 5 の $f(t)$ をサンプリング周波数 $f_s = 10$ [Hz]で $\textrm{T}_s = 4$ 秒間サンプリングして得られる時間領域ディジタル信号 $f[i]$ のグラフを描いてみましょう。\par
\noindent グラフは点あり、線あり、平滑化ありとします。

%%%%%%%%%%%%%%%%%%%%%%%%
\pracpage{標本化定理}

\pracnote{5} ある時間領域アナログ信号 $f(t)$ を $f_s = 10$ [Hz] でサンプリングした時のナイキスト周波数は何 [Hz] になるかノート上で計算して下さい。

\pracnote{5} ある時間領域アナログ信号 $f(t)$ を $f_s = 5$ [Hz] でサンプリングした時のナイキスト周波数は何 [Hz] になるかノート上で計算して下さい。

%%%%%%%%%%%%%%%%%%%%%%%%
\pracpage{エイリアシング}

\pracpc{10} 表計算ソフト使って、演習 5 の $f(t)$ をサンプリング周波数 $f_s = 5$ [Hz]で $\textrm{T}_s = 4$ 秒間サンプリングした時に得られる時間領域ディジタル信号 $f[i]$ のグラフを描いてみましょう。\par
\noindent グラフは点あり、線あり、平滑化ありとします。

\practeam{10} 演習 5 で描いた元のアナログ信号 $f(t)$ のグラフと、演習 6 と 演習 9 で描いたディジタル信号のグラフを見比べて、形の違いが生じた理由を話し合ってホワイトボードにまとめて下さい。\par
\noindent (ヒント) $f(t)$ は $2$ [Hz]のサイン波と $4$ [Hz]のサイン波を合成した信号

\savepractime

\end{document}
