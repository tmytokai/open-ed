\documentclass[a4paper]{jarticle}
\makeatletter 

\pagestyle{plain}

\parindent = 1zw
\hoffset = -1in
\voffset = -1in
\topmargin = 15mm
\headheight = 0mm
\headsep = 0mm
\oddsidemargin = 15mm
\evensidemargin = 15mm
\textwidth = 210mm  
\textheight = 287mm
\addtolength{\textwidth}{-\oddsidemargin}
\addtolength{\textwidth}{-\evensidemargin}
\addtolength{\textheight}{-\topmargin}
\addtolength{\textheight}{-\headheight}
\addtolength{\textheight}{-\headsep}
\addtolength{\textheight}{-\footskip}

%%%% 演習ページ用マクロ %%%%

% ページ番号
\newcounter{c-page}
\setcounter{c-page}{0}

% 課題番号
\newcounter{c-prac}
\setcounter{c-prac}{0}

% 合計時間
\newcounter{practime}
\setcounter{practime}{0}

\newcommand{\prachead}[2]{%
\noindent アクティビティ: #1\par%
\noindent 学習項目: #2\par%
\noindent 演習時間: \ref{label:practime} 分\par%
\vspace{2zh}%
\noindent \textbf{\Large 演習}%
}

\newcommand{\pracpage}[1]{%
\stepcounter{c-page}%
\vspace{4zh}%
\noindent%
\textbf{\noindent \textbf{\Large \arabic{c-page}. #1}}%
}

\newcommand{\practeam}[1]{%
\stepcounter{c-prac}%
\vspace{2zh}%
\noindent%
\textbf{演習 \arabic{c-prac} (チーム, #1 分):}%
\advance \c@practime by #1%
}

\newcommand{\pracnote}[1]{%
\stepcounter{c-prac}%
\vspace{2zh}%
\noindent%
\textbf{演習 \arabic{c-prac} (個人ノート, #1 分):}%
\advance \c@practime by #1%
}

\newcommand{\pracpc}[1]{%
\stepcounter{c-prac}%
\vspace{2zh}%
\noindent%
\textbf{演習 \arabic{c-prac} (個人PC, #1 分):}%
\advance \c@practime by #1%
}

\newcommand{\savepractime}{%
\addtocounter{practime}{-1}%
\refstepcounter{practime}%
\label{label:practime}%
}

\makeatother 


\begin{document}

\prachead{ディジタル信号処理の基礎}{[1] 時間領域ディジタル信号の定義とグラフ}

\pracpage{ディジタル信号}

\practeam{10} 世の中にあるディジタル信号 $f[i]$ の例を調べてみましょう。\par
\noindent 各自で調べ、結果を出しあってノートにまとめて下さい(1人1件)。

\pracpage{時間領域ディジタル信号}

\vspace{2zh}
\noindent 演習無し

\pracpage{時間領域ディジタル信号のグラフの描き方}

\pracnote{10} 手描きでノートに時間領域ディジタル信号 $f[i] = i^2$ のグラフを描いてみましょう。\par
\noindent 時刻の範囲は $i = 0,1,2,3$ とします。

\pracnote{10} 手描きでノートに時間領域ディジタル信号 $f[i] = 1$ のグラフを描いてみましょう。\par
\noindent 時刻の範囲は $i = 0,1,2,3$ とします。

\pracnote{10} 手描きでノートに直線 $i = 1.5$ のグラフを描いてみましょう。\par
\noindent 時刻の範囲は $i = 0,1,2,3$ とします。

\pracpc{10} 表計算ソフトで時間領域ディジタル信号 $f[i] = i^2$ のグラフを描いてみましょう。\par
\noindent 時刻の範囲は $i = 0,1,2,3$ とします。\par
\noindent グラフは点のみとします。

\pracpc{10} 表計算ソフトで時間領域ディジタル信号 $f[i] = 1$ のグラフを描いてみましょう。\par
\noindent 時刻の範囲は $i = 0,1,2,3$ とします。\par
\noindent グラフは点のみとします。

\pracpc{10} 表計算ソフトで直線 $i = 1.5$ のグラフを描いてみましょう。\par
\noindent 時刻の範囲は $i = 0,1,2,3$ とします。\par
\noindent (ヒント) $f[i] = 0$ のグラフを描いて描画機能で線を追加

\savepractime

\end{document}
