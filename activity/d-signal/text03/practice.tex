\documentclass[a4paper]{jarticle}
\makeatletter 

\pagestyle{plain}

\parindent = 1zw
\hoffset = -1in
\voffset = -1in
\topmargin = 15mm
\headheight = 0mm
\headsep = 0mm
\oddsidemargin = 15mm
\evensidemargin = 15mm
\textwidth = 210mm  
\textheight = 287mm
\addtolength{\textwidth}{-\oddsidemargin}
\addtolength{\textwidth}{-\evensidemargin}
\addtolength{\textheight}{-\topmargin}
\addtolength{\textheight}{-\headheight}
\addtolength{\textheight}{-\headsep}
\addtolength{\textheight}{-\footskip}

%%%% 演習ページ用マクロ %%%%

% ページ番号
\newcounter{c-page}
\setcounter{c-page}{0}

% 課題番号
\newcounter{c-prac}
\setcounter{c-prac}{0}

% 合計時間
\newcounter{practime}
\setcounter{practime}{0}

\newcommand{\prachead}[2]{%
\noindent アクティビティ: #1\par%
\noindent 学習項目: #2\par%
\noindent 演習時間: \ref{label:practime} 分\par%
\vspace{2zh}%
\noindent \textbf{\Large 演習}%
}

\newcommand{\pracpage}[1]{%
\stepcounter{c-page}%
\vspace{4zh}%
\noindent%
\textbf{\noindent \textbf{\Large \arabic{c-page}. #1}}%
}

\newcommand{\practeam}[1]{%
\stepcounter{c-prac}%
\vspace{2zh}%
\noindent%
\textbf{演習 \arabic{c-prac} (チーム, #1 分):}%
\advance \c@practime by #1%
}

\newcommand{\pracnote}[1]{%
\stepcounter{c-prac}%
\vspace{2zh}%
\noindent%
\textbf{演習 \arabic{c-prac} (個人ノート, #1 分):}%
\advance \c@practime by #1%
}

\newcommand{\pracpc}[1]{%
\stepcounter{c-prac}%
\vspace{2zh}%
\noindent%
\textbf{演習 \arabic{c-prac} (個人PC, #1 分):}%
\advance \c@practime by #1%
}

\newcommand{\savepractime}{%
\addtocounter{practime}{-1}%
\refstepcounter{practime}%
\label{label:practime}%
}

\makeatother 


\begin{document}

\prachead{ディジタル信号処理の基礎}{[3] 量子化}

%%%%%%%%%%%%%%%%%%%%%%%%
\pracpage{線形量子化}

\pracnote{10} テキストの図 2 を参考にして $f[i] = i^2/5$を $\Delta = 0.5$ で線形量子化している途中のグラフをノート上に描いてみましょう。\par
\noindent $i$ の範囲は $i = 0,1,\cdots,5$ とします。

\pracnote{10} その続きとして、テキストの図 3 を参考にして線形量子化後の $f[i]$ のグラフをノート上に描いてみましょう。\par
\noindent $i$ の範囲は同様に $i = 0,1,\cdots,5$ とします。

\practeam{10} (線形、非線形問わず)量子化幅 $\Delta$ を大きくした時のメリットとデメリットを話し合ってホワイトボードにまとめて下さい。\par
\noindent ※ 小さくしたときはメリットとデメリットが逆になります。

\bigskip
 
%%%%%%%%%%%%%%%%%%%%%%%%
\pracpage{(線形)量子化ビット数}

\practeam{10} CD、ハイレゾ、DSD(Direct Stream Digital)、VoLTE(AMR-WB) の量子化ビット数 $q$ [bit] を調べましょう。\par
\noindent 役割を分担して各自で調べ、結果を出しあってホワイトボードにまとめて下さい。

\pracnote{5} あるディジタル信号 $f[i]$ の値域を(0, 127.5)、量子化幅 $\Delta = 0.5$ とした時、値域の分割数はいくつになるかノート上で計算して下さい。

\pracnote{5} この時の量子化ビット数 $q$ はいくつなのか、ノート上で計算して下さい。 \par
\noindent (ヒント) $2^q-1=$ ? 等分より $q = $ ? [bit] 

\pracnote{10} この量子化を行なった後で、C言語を使って対象のディジタル信号列を記録したい場合は何型変数の配列を使えば良いのかノートに記して下さい。

\savepractime

\end{document}
