\documentclass[a4paper]{jarticle}

\def\texdir{../../../tex}
\makeatletter 

\pagestyle{plain}

\parindent = 1zw
\hoffset = -1in
\voffset = -1in
\topmargin = 15mm
\headheight = 0mm
\headsep = 0mm
\oddsidemargin = 15mm
\evensidemargin = 15mm
\textwidth = 210mm  
\textheight = 287mm
\addtolength{\textwidth}{-\oddsidemargin}
\addtolength{\textwidth}{-\evensidemargin}
\addtolength{\textheight}{-\topmargin}
\addtolength{\textheight}{-\headheight}
\addtolength{\textheight}{-\headsep}
\addtolength{\textheight}{-\footskip}

%% enumerate はカッコ付き数字
\renewcommand{\labelenumi}{(\arabic{enumi})}

%%%% 演習ページ用マクロ %%%%

% ページ番号
\newcounter{c-page}
\setcounter{c-page}{0}

% 課題番号
\newcounter{c-prac}
\setcounter{c-prac}{0}

% 合計時間
\newcounter{practime}
\setcounter{practime}{0}

\newcommand{\prachead}[2]{%
\noindent アクティビティ: #1\par%
\noindent 学習項目: #2\par%
\noindent 演習時間: \ref{label:practime} 分\par%
\vspace{2zh}%
\noindent \textbf{\Large 演習}%
}

\newcommand{\pracpage}[1]{%
\stepcounter{c-page}%
\vspace{4zh}%
\noindent%
\textbf{\noindent \textbf{\Large \arabic{c-page}. #1}}%
}

\newcommand{\practeam}[1]{%
\stepcounter{c-prac}%
\vspace{2zh}%
\noindent%
\textbf{演習 \arabic{c-prac} (チーム, #1 分):}%
\advance \c@practime by #1%
}

\newcommand{\pracnote}[1]{%
\stepcounter{c-prac}%
\vspace{2zh}%
\noindent%
\textbf{演習 \arabic{c-prac} (個人ノート, #1 分):}%
\advance \c@practime by #1%
}

\newcommand{\pracpc}[1]{%
\stepcounter{c-prac}%
\vspace{2zh}%
\noindent%
\textbf{演習 \arabic{c-prac} (個人PC, #1 分):}%
\advance \c@practime by #1%
}

\newcommand{\savepractime}{%
\addtocounter{practime}{-1}%
\refstepcounter{practime}%
\label{label:practime}%
}

\makeatother 


\begin{document}

\prachead{ディジタルサイン波}{[1] 時間領域ディジタルサイン波}

%%%%%%%%%%%%%%%%%%%%%%%%
\pracpage{時間領域ディジタルサイン波}


\pracpc{10} 表計算ソフトを使って時間領域ディジタルサイン波のグラフを描いてみましょう。\par
\vspace{1zh}
\begin{itemize}
\item 時間領域ディジタルサイン波の式は $f[i] = 2 \cdot \sin( 2\pi / 5 \cdot i )$ とします。
\item 時刻の範囲は $i = 0, 1,  \cdots, 20$ とします。
\item グラフは点あり、線あり、平滑化ありとします。
\end{itemize}

%%%%%%%%%%%%%%%%%%%%%%%%
\pracpage{振幅}

\vspace{2zh}
\noindent 演習無し

%%%%%%%%%%%%%%%%%%%%%%%%
\pracpage{角周波数、周波数、周期}

\pracnote{10} ディジタルサイン波の式から各パラメーターを求めてみましょう。
\par\vspace{1zh}
\begin{itemize}
\item 時間領域ディジタルサイン波の式は問題 1 と同様に $f[i] = 2 \cdot \sin( 2\pi / 5 \cdot i )$ とします。
\item サンプリング周波数は $f_s = 10$ [Hz] とします。
\item 振幅 $a$ 、周期 $\textrm{T}_d$ [点]、周波数 $f$ [Hz]、角周波数 $w$ [rad/秒]をノート上で計算してみましょう。
\end{itemize}

\pracnote{10} ディジタルサイン波の周波数と周期の関係を確認しましょう。
\par\vspace{1zh}
\begin{itemize}
\item サンプリング周波数は $f_s = 10$ [Hz] のままとします。
\item 周波数が $f = 1/2$ [Hz]の時、周期 $\textrm{T}_d$ は何点になるかノート上で計算してみましょう。
\end{itemize}

\pracpc{10} 表計算ソフトを使って、周波数を変えた時の時間領域ディジタルサイン波のグラフの変化を確認しましょう。
\par\vspace{1zh}
\begin{itemize}
\item 問題 1 のディジタルサイン波の周波数を $f = 1/2$ [Hz] に変えた時のグラフを描いて下さい。
\item 時刻の範囲は $i = 0, 1,  \cdots, 20$ とします。
\item グラフは点あり、線あり、平滑化ありとします。
\item グラフを描いたら問題 1 で描いたグラフとの違いを確認して下さい。
\end{itemize}

\newpage
%%%%%%%%%%%%%%%%%%%%%%%%
\pracpage{初期位相と進み・遅れ}

\pracnote{10} 初期位相と進み・遅れの関係について学びましょう。
\par\vspace{1zh}
\begin{itemize}
\item 時間領域ディジタルサイン波の式は $f[i] = 2 \cdot \sin( 2\pi / 5 \cdot i - 2\pi/5)$ とします。
※ 周期(周波数)が問題 1 のサイン波と同じであることに注意して下さい。
\item 初期位相 $\phi$ [rad] をノートに記述して下さい。
\item 問題1のディジタルサイン波(位相 = $0$)と比べて 何 [rad] 進んでいるのか、あるいは遅れているのかノートに記述して下さい。
\item 点を使って遅れ、進みをノートに記述して下さい。
\item グラフの平行移動方向(左か右か)と距離 [点]をノートに記述して下さい。
\end{itemize}

\pracpc{10} 表計算ソフトを使って、初期位相とグラフの平行移動の関係について学びましょう。
\par\vspace{1zh}
\begin{itemize}
\item サイン波の式は前問と同じく $f[i] = 2 \cdot \sin( 2\pi / 5 \cdot i - 2\pi/5)$ とします。
\item 時刻の範囲は $i = 0, 1,  \cdots, 20$ とします。
\item グラフは点あり、線あり、平滑化ありとします。
\item グラフを描いたら演習 1 で描いたグラフと比べて、前問で理論的に求めた並行移動方向と距離が合っていることを確かめて下さい。
\end{itemize}

%%%%%%%%%%%%%%%%%%%%%%%%
\pracpage{位相反転}

\pracpc{10} 表計算ソフトを使って、位相反転の効果について学びましょう。
\par\vspace{1zh}
\begin{itemize}
\item サイン波の式は演習 1 の位相を反転させたものとします。
\item グラフを描いたら問題 1 で描いたグラフとの違いを確認して下さい。
\end{itemize}

%%%%%%%%%%%%%%%%%%%%%%%%
\pracpage{直流(DC)信号}

\pracpc{10} 表計算ソフトを使って、直流(DC)信号のグラフを描いてみましょう。
\par\vspace{1zh}
\begin{itemize}
\item 式は $f[i] = -2$ とします。
\item 時刻の範囲は $i = 0, 1,  \cdots, 20$ とします。
\item グラフは点のみとします。
\end{itemize}


%%%%%%%%%%%%%%%%%%%%%%%%
\pracpage{グラフの描き方}

\vspace{2zh}
\noindent 演習無し

\savepractime

\end{document}
